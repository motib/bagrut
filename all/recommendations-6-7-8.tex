% !TeX root = bagrut-all.tex

\selectlanguage{english}
\cleardoublepage
\selectlanguage{hebrew}

\section*{המלצות: חשבון דיפרנציאלי ואינטגרלי}

\addcontentsline{cot}{chapter}{המלצות: חשבון דיפרנציאלי ואינטגרלי}

\begin{itemize}
\item
אי-אפשר להכין טבלה של עליות וירידות עד שלא מחשבים את תחום ההגדרה 
\textbf{וגם}
כל נקודות הקיצון של הפנוקציה, כי רק ביניהם אפשר לסמוך על זה שאין שינוי בכיוון הפונקציה.


\item
אני אוהב להשתמש בתנועות ידיים כדי "לראות שיפועים" ולקבוע אם פונקציה עולה או יורדת, וכן אם נקודות קיצון היא מקסימום או מינימים. אני מזיז כף יד שטוחה לאורך הפונקציה מערכים שליליים לחיוביים על ציר ה-%
$x$.
אם כיוון היד למטה הפונקציה יורדת, ואם הכיוון למעלה הפונקציה עולה. אם היד עוברת מכיוון למטה לכיוון למעלה, השיפוע )הנגזרת( עולה, כך שהנגזרת השנייה היא חיובית ונקודת הקיצון היא מינימים. שינוי הפוך בכיוון היד מראה שקיים מקסימום.

\item
אני מציע להתרחק מהמחשבון עד כמה שאפשר ולחשב עם סימנים אלגבריים. הסיבה היא שקשה למצוא שגיאות הנגרמות מטעויות בקלדה על המחשבון, אבל אפשר לעבור שוב ושוב על חישוב אלגבראי כדי לוודא את נכונותו. 

\item
אל תקצרו בחישובים. לעתים קרובות שגיתי כי השמטתי סימן מינוס. לוקח מעט זמן לרשום שורה נוספת לעומת הזמן הדרוש לחפש שגיאה בחישוב מקוצר.

\item
אני מעדיף לסווג נקודות קיצון על ידי בדיקת הסימן של הנגזרת השנייה ולא על ידי חישוב טבלת עליות וירידות. 

\item
כלום יודעים שאם הסימן של המכנה של הנגזרת הראשונה חיובי, הסימן של הנגזרת השנייה זהה לסימן של הנגזרת של המונה של הנגזרת הראשונה. חשוב לא לטעון שהנגזרת השנייה
\textbf{שווה}
לנגזרת של המונה של הנגזרת הראשונה!%
\footnote{%
ראו יונתן אחיטוב. 'גזירה שנייה מקוצרת' כדרך לאפיון נקודות קיצון. על"ה 
$20$,
$2002$,
עמ'
$26\!-\!27$.%
}
עם זאת, יש נוסחה לנגזרת השנייה התקפה 
\textbf{רק עבור נקודות בהן הנגזרת הראשונה מתאפסת והן חשודות כנקודות הקיצון}.%
\footnote{%
ראו במאמר של אחיטוב וגם "ללמוד וללמד אנליזה", עמ'
$322-323$.%
}
עבור:
\[
f'(x)=\frac{g(x)}{h(x)}
\]
ו-%
$a$
כך ש-%
$f'(a)=0$:
\[
f''(a) = \frac{g'(a)}{h(a)}\,.
\]
ההוכחה פשוטה: חישבו את הנגזרת. נביא דוגמה:
\erh{14pt}
\begin{equationarray*}{rcl}
f(x)&=&\frac{x^2+x-1}{x^2-x+1}\\
f'(x)&=&\frac{-2x^2+4x}{(x^2-x+1)^2}\,.
\end{equationarray*}

\vspace{-4ex}

נקודות הקיצון הן:
\[
a_1=(0,-1),\;a_2=\left(2,\frac{5}{3}\right)\,.
\]

\np

בגלל שהמכנה חיובי, הסימן של הנגזרת השנייה הוא הסימן של:
\[
\left(-2x^2+4x\right)'=-4x+4\,,
\]
למרות שזו לא הנגזרת השנייה. עבור נקודות הקיצון,
$-4\cdot 0 + 4=4>0$
ו-%
$a_1$
היא מינימום, ו-%
$-4\cdot 2+4=-4<0$
ו-%
$a_2$
היא מקסימום.

עבור הנקודות
$a=a_1,a=a_2$
אפשר לחשב את הנגזרת השנייה ולבדוק ש:
\[
f''(a)=\frac{-4a+4}{a^2-a+1}\,.
\]

\vspace{-4ex}

\item 
שימו לב להגדרה של נקודת פיתול: נקודה בה משתנה הקעירות של הפונקציה. בנקודת פיתול, הנגזרת השנייה מתאפסת או לא מוגדרת, אבל יש פונקציות שמקיימות אחד מהתנאים הללו בנקודה מסויימת אבל אין שם נקודת פיתול. למשל, הנגזרת שנייה של
$f(x)=x^4$
מתאפסת ב-%
$x=0$
אבל יש שם מינימום ולא נקודת פיתול.%
\footnote{%
"ללמוד וללמד אנליזה", עמ'
$227$.}

בנקודת פיתול 
\textbf{הנגזרת הראשונה}
לא חייב להתאפס. למשל, עבור
$f(x)=\sin x$:
\erh{1pt}
\begin{equationarray*}{rcl}
f'(0) &=& \cos 0=1\\
f''(0) &=& -\sin 0 = 0\,.
\end{equationarray*}

\vspace{-6ex}
\item
כאשר מופיע שורש בפונקציה הכוונה היא לשורש החיובי. אבל, כאשר לוקחים שורש של
$x^2$
הכוונה היא לערך המוחלט של
$x$.
למשל, אם 
$x=3$,
$\sqrt{3^2}= 3 = x$,
אבל אם 
$x=-3$,
$\sqrt{(-3)^2}= 3 = -x$.
זה חשוב כאשר
\asms{}
של פונקציות עם שורשים:
\[
f(x)=\frac{x}{\sqrt{x^2-a^2}}=\frac{\disfrac{x}{\sqrt{x^2}}}{\sqrt{1-\disfrac{a^2}{x^2}}}=\disfrac{\disfrac{x}{|x|}}{\sqrt{1-\frac{a^2}{x^2}}}\limit{\pm\infty}\pm 1\,.
\]

\vspace{-4ex}

\item
אם השאלה מבקשת נקודות חיתוך עם הצירים או נקודות קיצון, יש נטייה להסתפק בחישב ערך ה-%
$x$,
אבל התשובה חייבת להיות קואורדינטות
$(x,y)$.

\item
כאשר מבקשים לחשב אינטגרל "מפחיד" של פונקציה, תמיד הפונקציה תהיה נגזרת של פונקציה שאפשר לנחש בקלות. למשל:
\[
\int\frac{2\sin x}{\cos^3 x}dx
\]
נראה "מפחיד" אבל אם נתבונן בו קצת נבין ש:
\[
(\cos^{-2} x)=-2\cdot (-\sin x) \cos^{-3}=\frac{2\sin x}{\cos^3 x}\,.
\]
לפעמים מבקשים
$\displaystyle\int f'(x) dx$
כאשר 
$f(x)$
נתון, ואז אין מה לחשב!

\np

\item 
שימו לב להבדל בין שטח לאיטגרל.

\begin{center}
\selectlanguage{english}
\begin{tikzpicture}[scale=.8]
\begin{axis}[
    trig format plots=rad,
    axis lines=center,
    xmin = 0,
    xmax = 7,
    ymin = -1.1,
    ymax = 1.1,
    xtick={0,3.14,6.28},
    xticklabels={$0$, $\pi$, $2\pi$},
    xticklabel style={anchor=south west},
]
\addplot [
    domain=0:6.28,
    samples=60, 
]
{sin(x)};
\end{axis}
\end{tikzpicture}
\end{center}

אם מבקשים את האיטגרל של סינוס מאפס עד 
$2\pi$,
התשובה היא אפס:
\[
\int_0^{2\pi} \sin x\, dx =-\left.\cos x\right|_0^{2\pi}= -(1-1)=0\,.
\]

\vspace{-4ex}

אבל אם מבקשים את השטח התחום על ידי הפונקציה וציר ה-%
$x$
התשובה היא:
\[
\int_0^{\pi} \sin x\, dx + \int_{\pi}^{2\pi} -\sin x\, dx = -\left.\cos x\right|_0^{\pi} \left.+\cos x\right|_{\pi}^{2\pi}= -(-1-(1))+(1-(-1))=4\,.
\]

\vspace{-4ex}


\item
איטגרל של פונקציה אי-זוגית בתחום סימטרי סביב ציר ה-%
$y$
הוא אפס, ואינטגרל של פונקציה זוגית בתחום סימטרי הוא פי שניים האינטגרל של התחום החיובי בלבד.


\item
ראו נספח~%
\ref{a.unit-circle}
המסביר את החשיבות של מעגל היחידה בחישובים טריגונומטריים.


\item
בבעיות עם פונקציות טריגונומטריות, בנו תרשים של מעגל היחידה וסמנו עליו את תחום ההגדרה. התרשים יעזור בקביעת סימני הפונקציות ובערכי הפונקציות כאשר מוסיפים או מחסירים כפולות רציונליות של 
$\pi$.

\begin{center}
\begin{minipage}{.45\textwidth}
\selectlanguage{english}
\begin{tikzpicture}[scale=.85]
\coordinate (O) at (0,0);
\coordinate (A) at (0,-2);
\node[draw,circle through=(A)] at (O) {};
\draw (-2,0) -- (2,0);
\draw (0,-2) -- (0,2);
\draw[ultra thick] (A) arc[start angle=-90,end angle=180,radius=2];
\node[right] at (2,0) {$0$};
\fill ($(O)+(-90:2)$) circle(2pt) node[below] {$-\pi/2$};
\fill ($(O)+(90:2)$) circle(2pt) node[above] {$\pi/2$};
\fill ($(O)+(180:2)$) circle(2pt) node[left] {$\pi$};
\draw[thick,dashed] (O) -- (30:2) node[above right] {$\pi/6$};
\draw[thick,dashed] (O) -- (-60:2) node[below right] {$-\pi/3$};
\draw[thick,dotted] (O) -- (150:2) node[above left] {$5\pi/6$};
\draw[thick,dotted] (O) -- (60:2) node[above right] {$\pi/3$};
\node at (0,4.6) {$(5\pi/6)-(\pi/2)=(\pi/3)$};
\node at (0,3.6) {$(\pi/6)-(\pi/2)=-(\pi/3)$};
\end{tikzpicture}
\end{minipage}
\begin{minipage}{.45\textwidth}
\selectlanguage{english}
\begin{tikzpicture}[scale=.85]
\coordinate (O) at (0,0);
\coordinate (A) at (2,0);
\node[draw,circle through=(A)] at (O) {};
\draw (-2,0) -- (2,0);
\draw (0,-2) -- (0,2);
\draw[ultra thick] (A) arc[start angle=0,end angle=-90,radius=2];
\node[right] at (2,0) {$0$};
\fill ($(O)+(-90:2)$) circle(2pt) node[below,xshift=-6pt] {$-\pi/2$};
\fill ($(O)+(0:2)$) circle(2pt) node[right] {$0$};
\draw[thick,dashed] (O) -- (-15:2) node[right] {$-\pi/12$};
\draw[very thick,dotted] (O) -- (-75:2) node[below right] {$-5\pi/12$};
\draw[thick,dashed] (O) -- (-30:2) node[below right] {$-\pi/6$};
\draw[very thick,dotted] (O) -- (-150:2) node[below left] {$-5\pi/6$};
\draw[very thick,dotted] (-2.9,-1) -- +(6,0);
\fill ($(O)+(-15:2)$) circle(2pt);
\fill ($(O)+(-30:2)$) circle(2pt);
\fill ($(O)+(-75:2)$) circle(2pt);
\fill ($(O)+(-150:2)$) circle(2pt);
\node[above] at ($(O)+(90:2)$) {\rule{0pt}{15pt}};
\node at (0,4.6) {$\sin 2x=-(1/2)$};
\node at (0,3.6) {$2x=-(5\pi/6),\; x=-(5\pi/12)$};
\node at (0,2.6) {$2x=-(\pi/6),\; x=-(\pi/12)$};
\end{tikzpicture}
\end{minipage}
\end{center}

\item
נניח שהתחום הנתון הוא
$0\leq x \leq \frac{\pi}{2}$,
ונניח שמקבלים את התוצאה
$\sin 2x=\frac{\sqrt{3}}{2}$.
ברור שתשובה אחת היא
$2x=\frac{\pi}{3}$
ו-%
$x=\frac{\pi}{6}$.
לא לשכוח את התשובה השנייה: 
$\sin 2\left(\frac{\pi}{3}\right)=\frac{\sqrt{3}}{2}$,
ו-%
$x=\frac{\pi}{3}$
בתחום, למרות ש-%
$2x=\frac{2\pi}{3}$
אינו בתחום.

\np

\item
בנוסחאון נתון:
\[
\textrm{(\R{ממשי} t)} \quad\quad (x^t)' = tx^{t-1}\,,
\]
אולם משתמשים בנוסחה רק עבור 
$t$
שלם וחיובי. אני מעדיף להשתמש בנוסחה זו עבור
\textbf{כל}
$t$,
כי קל לזכור את הנוסחה והחישובים פשוטים. למשל, מיותר לזכור את הנוסחה הנתונה:
\[
(\sqrt{x})' = \disfrac{1}{2\sqrt{x}}
\]
כי:
\[
(\sqrt{x})' = (x^\frac{1}{2})'= \frac{1}{2}(x^{-\frac{1}{2}}) = \frac{1}{2}\cdot\frac{1}{x^{\frac{1}{2}}}=\disfrac{1}{2\sqrt{x}}\,.
\]
החיסכון בולט כאשר צריכים לחשב נגזרת רציונלית. כאשר
$f(x)=1, g(x)=x^t$,
במקום להשתמש בנוסחה המוסבכת עבור הנגזרת של
$\frac{f(x)}{g(x)}$,
פשוט יותר לחשב:
\[
\left(\frac{1}{x^t}\right)'=(x^{-t})'=-t(x^{-t-1})=\frac{-t}{x^{t+1}}\,.
\]
כאשר במונה יש קבוע ובמכנה יש פונקציה מורכבת עדיין החישוב לא מסובך. למשל:
\erh{14pt}
\begin{equationarray*}{rcl}
\left(\frac{14}{x^2-3x+4}\right)'&=&14\left((x^2-3x+4)^{-1}\right)'\\
&=&-14(x^2-3x+4)^{-2}(2x-3)\\
&=&\frac{-14(2x-3)}{(x^2-3x+4)^{2}}\,.
\end{equationarray*}

\item
אם
$f(x)=0$
אז
$-f(x)=-0=0$.
אמנם התוצאה פשוטה אבל היא שימושית כאשר חושבו נקודות איפוס של נגזרת, וצריך למצוא נקודות איפוס של השלילה של הנגזרת, למשל:
\[
(g(x)-f(x))' = (-1\cdot (f(x)-g(x))' = -1 (f(x)-g(x))'\,,
\]
ולכן אם 
$(f(x_1)-g(x_1))'=0$,
מתקבל מייד
$(g(x_1)-f(x_1))'=-1\cdot 0 = 0$.

\end{itemize}

\npchap
