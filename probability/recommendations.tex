% !TeX root = probability.tex

%%%%%%%%%%%%%%%%%%%%%%%%%%%%%%%%%%%%%%%%%%%%%%%%%%%%%%%%

\section*{המלצות}

\addcontentsline{toc}{section}{\large המלצות}

\begin{itemize}
\item
קרא בזהירות את השאלות. לעתים הן ארוכות וחשוב להבין את המשמעות של כל פסקה.

\item
כמעט כל הבחינות מכילות שאלות על הסתברות מותנית. ניסוחים רבים מכוונים להסתברות מותנית וחשוב להכיר אותם!

\begin{itemize}
\item
הניסוח השכיח ביותר משתמש במילים
"\textbf{אם ידוע ש-}"
או
"\textbf{ידוע כי}".

\item
בבחינה של חורף תשע"ז כתוב "%
\textbf{אם} $\ldots$ ,
\textbf{מהי ההסתברות} $\ldots$".
לא לגמרי ברור שלמילה "אם" יש משמעות של "אם ידוע", אבל זאת הכוונה.

\item
לעתים קרובות )בחינה של קיץ תשע"ה ב'( כתוב "%
\textbf{מה ההסתברות לבחור} $\ldots$
\textbf{מבין} $\ldots$".

\item
יוצא מן הכלל: בבחינה של קיץ תשע"ו א' כתוב
"\textbf{מבין}
כל הנבחנים". המילה "מבין" בדרך כלל מכוונת להסתברות מותנית, אבל כאשר "מבין" מתייחס ל-%
"\textbf{כל}
הנבחנים", אין הסתברות מותנית. לחילופין אפשר לחשב הסתברות מותנית כאשר החיתוך מצטמצם:
\[
P(X/\textrm{\R{כל הנבחנים}})=
\frac{P(X\cap \textrm{\R{כל הנבחנים}})}
{P(\textrm{\R{כל הנבחנים}})} = 
\frac{P(X)}{1}=P(X)\,.
\]

\item
בבחינה של קיץ תשע"ח א' הניסוח הוא: "%
$n\%$
נעזרו בחבריהם 
$N$
ו-%
$\displaystyle\frac{k}{n}$
\textbf{מהם}
עברו את הבחינה
$A$".
ברור ש-%
$P(A\cap N) = k$,
אבל נבדוק לפי הנוסחה להסתברות מונתית:
\begin{eqnarray*}
P(A/N) &=& \frac{P(A\cap N)}{P(N)} = \frac{k}{n}\\
P(A\cap N)&=&k\,.
\end{eqnarray*}

\item
בבחינה של חורף תשע"ד יש ניסוח אחר: "כל התושבים המשתתפים ב-
$\ldots$,
\textbf{ורק הם}".
\end{itemize}

%%%%%%%%%%%%%%%%%%%%%%%%%%%%%%%%%%%%

\item
כאשר יש חיתוך בחישוב של הסתברות מותנית, לעתים קרובות ניתן לפשט את החישוב. בבחינה של קיץ תשע"ז א' יש לחשב
$P(D=4\cap D\ge 3)$,
אבל אם ערך גדול או שווה
$3$
\textbf{וגם}
שווה ל-%
$4$,
אז הוא שווה ל-%
$4$, 
ולכן מספיק לחשב
$P(D=4)$.

%%%%%%%%%%%%%%%%%%%%%%%%%%%%%%%%%%%%

\item
אם שני אירועים בלתי תלויים, חישוב ההסתברות המותנית מצטמצם:
\[
P(A/B) = \frac{P(A\cap B)}{P(B)} = \frac{P(B)\cdot P(A)}{P(A)}= P(B)\,.
\]
מצב זמ מופיע בבחינות של חורף תשע"ז, חורף משע"ח, קיץ תשע"ה א', חורף תשע"ד.
%%%%%%%%%%%%%%%%%%%%%%%%%%%%%%%%%%%%



\item
המילה 
\textbf{בדיוק}
מכוונת לחישוב אחד של נוסחת ברנולי, כי נתון כמה "הצלחות" צריכות להיות וגם כמה "כשלונות".

%%%%%%%%%%%%%%%%%%%%%%%%%%%%%%%%%%%%

\item
בבחינה של קיץ תשע"ז א' כתוב "%
\textbf{בוחרים באקראי}
$\ldots$,
\textbf{עד של-}
$3$
מהם
\textbf{בדיוק}
יש קלנועית". המשמעות של "עד ש-" היא שמפסיקים את הבחירה האקראית כאשר הבחירה 
\textbf{האחרונה} 
היא "הצלחה". במקרה זה נשארו שתי "הצלחות" שיש לחשב את ההסתברות שלהן לפי נוסחת ברנולי, ואז להכפיל בהסתברות של "הצלחה" בבחירה האחרונה:
\[
\overbrace{\pm\;\pm\;\pm\;\pm\;\pm}^{2/5}\quad\quad \overbrace{+}^{1/1}\,.
\]

%%%%%%%%%%%%%%%%%%%%%%%%%%%%%%%%%%%%

\item
בבחינה של קיץ תשע"ז ב' הביטוי "מוציאים באקראי
$\ldots$",
ובהמשך הביטוי "מוציאים באקראי
\textbf{שוב}
$\ldots$"
מכוון לשימוש בעץ כדי לתאר את הבחירה הסדרתית.

%%%%%%%%%%%%%%%%%%%%%%%%%%%%%%%%%%%%

\item
בבחינה של קיץ תשע"ח א' המשמעות של הניסוח "%
\textbf{לפחות אחת}
משתי הטענות I, II היא שהאירוע קורה אם קורה אחד מהאירועים I, II,
\textbf{או שניהם},
המסומן I
$\cup$
II".
יש שתי דרכים לחשב את ההסתברות:
\begin{eqnarray*}
P(\textrm{I} \cup \textrm{II}) &=& P(\textrm{I}) + P(\textrm{II}) - P(\textrm{I} \cap \textrm{II})\\
P(\textrm{I} \cup \textrm{II}) &=& P(\textrm{I}-\textrm{II}) + P(\textrm{II}-\textrm{I}) + P(\textrm{I} \cap \textrm{II})\,.
\end{eqnarray*}

%%%%%%%%%%%%%%%%%%%%%%%%%%%%%%%%%%%%

\item
בבחינה של  קיץ תשע"ח ב' יש לחשב את ההסבתרות לפי נוסחת ברנולי
${n \choose k}p^k(1-p)^{n-k}$.
\begin{itemize}
\item
אם
$k=0$
אזי
${n\choose 0}=1$
והנוסחה מצטמצמת ל-%
$p^0(1-p)^{n-0}=(1-p)^n$.
\item 
אם
$k=n$
אזי
${n\choose n}=1$
והנוסחה מצטמצמת ל-%
$p^n(1-p)^{n-n}=p^n$.
\end{itemize}


%%%%%%%%%%%%%%%%%%%%%%%%%%%%%%%%%%%%

\item
בבחינות של קיץ תשע"ו א' ו-ב' יש שלוש תוצאות לפעולה במקום שתיים. סכום ההסתברויות חייב להיות אחד, ולכן כאשר מחשבים משלים להסתברות אחת, יש להחסיר את שתי ההסתברויות האחרות. בבחינה של מועד ב' ההסתברות לתיקו היא אחד פחות ההסתברות שיעל תנצח פחות ההסתברות אנה תנצח:
\[
P(\textrm{\R{תיקו}}) =
1 - (P(\textrm{\R{יעל}})+
P(\textrm{\R{אנה}})) = 
1 - P(\textrm{\R{יעל}})-
P(\textrm{\R{אנה}}) \,.
\]

\item 
במספר בחינות (חורף תשע"ה, קיץ תשע"ד ב', קיץ תשע"ה ב') מתואר מצב הנקרא "שליפה ללא החזרה". אם יש מספר נמוך של תושבים, השליפות לא בלתי-תלויות. כאשר כתוב "ישוב גדול", "עיר גדולה", "אוניברסיטה גדולה", אני מניח שכוונה שיש מספר כל כך גדול של תושבים שאין שינוי משמעותי בהסתברות משליפה אחת לבאה אחריה.
\end{itemize}

