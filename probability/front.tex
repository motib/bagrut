% !TeX root = probability.tex

\selectlanguage{hebrew}

\thispagestyle{empty}

\begin{center}
\textbf{\LARGE בחינות בגרות בהסתברות}
\end{center}

\bigskip
\bigskip

\begin{center}
\textbf{\Large מוטי בן-ארי}

\bigskip

\url{http://www.weizmann.ac.il/sci-tea/benari/}
\end{center}

\begin{center}	
\begin{bfseries}
\bigskip
\bigskip

\R{גרסה} \L{1.0} 

\bigskip

\today

\end{bfseries}
\end{center}

\vfill

\selectlanguage{english}

\begin{small}
\begin{center}
\copyright{}\ 2022 \R{מוטי בן-ארי}
\end{center}

This work is licensed under a Creative Commons Attribution-ShareAlike 4.0 International License:
\url{http://creativecommons.org/licenses/by-sa/4.0/}.
\end{small}

\bigskip

\begin{center}
\includegraphics[width=.2\textwidth]{../by-sa.png}
\end{center}

\newpage

\selectlanguage{hebrew}

\thispagestyle{empty}

\tableofcontents

\newpage

\section*{מבוא}
\addcontentsline{toc}{section}{\large מבוא}

חוברת זו כוללת פתרונות לכל השאלות על הסתברות של בחינות הבגרת (שאלון 
$806 / 581$)
מהשנים תשע"ד עד תשפ"ב. 
הדגשים בפתרונות הם:
\begin{itemize}
\item 
זיהוי מוקפד של המאורעות.

\item
הנמקה של בחירת שיטות לחישוב ולהצגת החישוב: עץ, טבלה, ברנולי, בינום, עם דגש מיוחד על הבנת הניסוחים הרבים המכוונים להסתברות מותנית.
\item
לא הססתי לכלול תיאור של מקרים בהם הסתבכתי בפתרון!
\item
בסוף החוברת נמצא סעיף "המלצות" המסכם לקחים מהפתרונות.
\item
החוברת מופצת עם רישיון המאפשר העתקה חופשית. ניתן להוריד את המסמכים ב-%
\L{PDF}
מ:
\begin{center}
\selectlanguage{english}
\url{https://github.com/motib/bagrut}
\selectlanguage{hebrew}
\end{center}
שם נמצא גם קוד המקור ב-%
\L{\LaTeX{}}.
המסמך בעברית ויש להשתמש ב-%
\L{\XeLaTeX{}}
ולא ב-%
\L{pdflatex}.
\end{itemize}

\textbf{סימון המאורעות}
אני מקפיד עם סימון מאורעות כי לדעתי זה מקל על הבנת החישובים לעומת שימוש בשפה טבעית. הסימון גם מעודד חשיבה לזיהוי מוקפד של המאורעות. למשל:
\begin{quote}
נסמן ב-%
$N$ \L{(neta)}
ניצחון של נטע במשחק.
\end{quote}
כאשר מבקשים גם את מספר הנצחונות של נטע נשתמש בסימון כגון
$N=4$
או
$N\geq 5$.\footnote{$N\geq 5$
הוא למעשה משתנה אקראי שערכו גדול או שווה ל-$5$, אבל המונח לא נמצא בתכנית הלימודים.}

