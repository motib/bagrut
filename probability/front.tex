% !TeX root = probability.tex

\selectlanguage{hebrew}

\thispagestyle{empty}

\begin{center}
\textbf{\LARGE בחינות בגרות בהסתברות}
\end{center}

\bigskip
\bigskip

\begin{center}
\textbf{\Large מוטי בן-ארי}

\bigskip

\url{http://www.weizmann.ac.il/sci-tea/benari/}
\end{center}

\begin{center}	
\begin{bfseries}
\bigskip
\bigskip

\R{גרסה} \L{1.0} 

\bigskip

\today

\end{bfseries}
\end{center}

\vfill

\selectlanguage{english}

\begin{small}
\begin{center}
\copyright{}\ 2022 \R{מוטי בן-ארי}
\end{center}

This work is licensed under a Creative Commons Attribution-ShareAlike 4.0 International License:
\url{http://creativecommons.org/licenses/by-sa/4.0/}.
\end{small}

\bigskip

\begin{center}
\includegraphics[width=.2\textwidth]{../by-sa.png}
\end{center}

\newpage

\selectlanguage{hebrew}

\thispagestyle{empty}

\tableofcontents

\newpage

\section*{מבוא}
\addcontentsline{toc}{section}{\large מבוא}

חוברת זו מבוסס על הספר "בחינות בגרות במתמטיקה" כוללת פתרונות לכל השאלות בבחינות הבגרות שאלון
$806 / 581$
מהשנים תשע"ד עד תשע"ח. החוברת כולל רק פתרונות לשאלות בהסתברות שעברו שיכתוב כדי לשפר הצגת בפרתונות. כמו כן, יתווספו פתרונות לשאלות מהשנים האחרונות. הדגשים בפתרונות הם:
\begin{itemize}
\item 
זיהוי מוקפד של המאורעות.
\item
סימון תואם את השאלה כדי להקל על הקשר בין השאלה והחישובים, למשל, 
$T,R$
כדי לסמן משתתפים בחוג תיאטרון וחוג ריקוד במקום 
$A,B$.
\item
הנמקה של בחירת שיטות לחישוב ולהצגת החישוב: עץ, טבלה, ברנולי, הסתברות מותנית. במיוחד חשוב ללמוד לפרש את הניסוחים הרבים המכוונים להסתברות מותנית.
\end{itemize}
בסוף החוברת נמצא סעיף "המלצות" המסכם לקחים מהפתרונות.

החוברת מופצת עם רישיון המאפשר העתקה חופשית וכן הכנת גרסאות חדשות ובתנאי שנשמר שם המחבר והגרסאות מופצות עם אותו רישיון.

ניתן להוריד את המסמכים ב-%
\L{PDF}
וכן את קוד המקור ב-%
\L{\LaTeX{}}
מ:
\begin{center}
\selectlanguage{english}
\url{https://github.com/motib/bagrut}.
\selectlanguage{hebrew}
\end{center}
