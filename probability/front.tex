% !TeX root = probability.tex

\selectlanguage{hebrew}

\thispagestyle{empty}

\begin{center}
\textbf{\LARGE בחינות בגרות בהסתברות}
\end{center}

\bigskip
\bigskip

\begin{center}
\textbf{\Large מוטי בן-ארי}

\bigskip

\url{http://www.weizmann.ac.il/sci-tea/benari/}
\end{center}

\begin{center}	
\begin{bfseries}
\bigskip
\bigskip

\R{גרסה} \L{1.0} 

\bigskip

\today

\end{bfseries}
\end{center}

\vfill

\selectlanguage{english}

\begin{small}
\begin{center}
\copyright{}\ 2022 \R{מוטי בן-ארי}
\end{center}

This work is licensed under a Creative Commons Attribution-ShareAlike 4.0 International License:
\url{http://creativecommons.org/licenses/by-sa/4.0/}.
\end{small}

\bigskip

\begin{center}
\includegraphics[width=.2\textwidth]{../by-sa.png}
\end{center}

\newpage

\selectlanguage{hebrew}

\thispagestyle{empty}

\tableofcontents

\newpage

\section*{מבוא}
\addcontentsline{toc}{section}{\large מבוא}

חוברת זו מבוססת על החוברת "בחינות בגרות במתמטיקה" כוללת פתרונות לכל השאלות בבחינות הבגרות שאלון
$806 / 581$
מהשנים תשע"ד עד תשע"ח. חוברת זו כוללת פתרונות לשאלות בהסתברות מאותן שנים וכן מהשנים תשע"ט עד תשפ"ב. 

הדגשים בפתרונות הם:
\begin{itemize}
\item 
זיהוי מוקפד של המאורעות.
\item
סימון המאורעות כדי לפשט את החישובים, למשל, "נסמן ב-% 
$R$
את המשתתפים בחוג לריקוד. במקרה זה הסימון גם משמש למאורעות של כמות:
$P(R\geq 5)$
היא ההסתברות שמספר המשתתפים בחוג לריקוד גדול או שווה לחמשה.

\item
הנמקה של בחירת שיטות לחישוב ולהצגת החישוב: עץ, טבלה, ברנולי, הסתברות מותנית. אני מדגיש במיוחד את הבנת הניסוחים הרבים המכוונים להסתברות מותנית.
\item
לא הססתי לכלול תיאור של מקרים שהסתבכתי בפתרון!
\end{itemize}
בסוף החוברת נמצא סעיף "המלצות" המסכם לקחים מהפתרונות.

החוברת מופצת עם רישיון המאפשר העתקה חופשית.

ניתן להוריד את המסמכים ב-%
\L{PDF}
מ:
\begin{center}
\selectlanguage{english}
\url{https://github.com/motib/bagrut}
\selectlanguage{hebrew}
\end{center}
שם נמצא גם קוד המקור ב-%
\L{\LaTeX{}}.
המסמך בעברית ויש להשתמש ב-%
\L{\XeLaTeX{}}
כדי להפיק את ה-%
\L{PDF}.

